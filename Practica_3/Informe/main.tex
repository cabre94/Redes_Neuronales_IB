\documentclass[11pt,a4paper]{article}

\usepackage[utf8]{inputenc}
\usepackage{graphicx}
\usepackage[spanish]{babel}
\usepackage{float}				%Para poner las imagenes exactamente donde se me cante las pelotas en caso de quererlo, poniendole [H]
\usepackage{amsmath}
\usepackage{epstopdf}
\usepackage{geometry}
\usepackage{hieroglf}
\usepackage{subcaption}
\usepackage[justification=centering]{caption}
\usepackage[colorinlistoftodos]{todonotes}
\usepackage[colorlinks=true, allcolors=blue]{hyperref}
\geometry{
a4paper,
left=20mm,
right=20mm,
top=25mm,
bottom = 20mm
}
\usepackage{float}
\usepackage{units}
% \usepackage{hyperref}   %Esto es para ir a los links

\newcommand{\rojo}[1]{\textcolor{red}{#1}}    % Comando para escribir texto en rojo




\title{\textbf{Redes Neuronales \\ Práctica 3 - Estadística de trenes de spikes}}

\author{
{F. M. Cabrera}
%[1ex] \small{\textit{ Facultad de Ciencias Exactas y Naturales.}} \\
%\small{\textit{Universidad de Buenos Aires. Ciudad Universitaria. Pabellón I. Buenos Aires. Argentina}}
}
\date{\textit{\today}}


% Esto modifica el interlineado
\renewcommand{\baselinestretch}{1}

\graphicspath{{Figuras/}}

\begin{document}

\maketitle

%\thispagestyle{empty} 


\setcounter{page}{1}

%\begin{abstract}


%\end{abstract}
%\vspace*{1cm}

Se analizó estadísticamente los resultados de actividad neuronal de un saltamontes frente a un estímulo acústico. Los datos utilizados se corresponden a la envolvente de la onda sonora utilizada como estímulo y a la actividad neuronal del saltamontes durante 128 realizaciones distintas, de un segundo de duración y con 10000 ventanas de $0.1\,\text{ms}$. Todo el código implementado en esta practica puede encontrarse \href{https://github.com/cabre94/Redes_Neuronales_IB}{acá}.
%  \rojo{Sacar esto \cite{HH}}


\section*{Ejercicio 1}
\graphicspath{{/home/cabre/Desktop/Redes_Neuronales/Redes_Neuronales_IB/Practica_5/Figuras}}



\begin{figure}
    \centering
    \includegraphics[width=\textwidth]{../1.pdf}
    \caption{.}
    \label{fig:01}
\end{figure}
\section*{Ejercicio 2}
\graphicspath{{/home/cabre/Desktop/Redes_Neuronales/Redes_Neuronales_IB/Practica_5/Figuras/}}


\begin{figure}
    \centering
    \includegraphics[width=\textwidth]{../2.pdf}
    \caption{.}
    \label{fig:02}
\end{figure}


\section*{Ejercicio 3}
\graphicspath{{Figuras/}}


\section*{Ejercicio 4}
\graphicspath{{Figuras/}}





% \bibliographystyle{acm}
% \bibliography{biblio}

\end{document}



